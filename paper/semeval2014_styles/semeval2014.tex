%
% File twocolumn.tex
%
%%
%% Based on the style files for *SEM-2014, which were, in turn,
%% Based on the style files for COLING-2014, which were, in turn,
%% Based on the style files for ACL-2014, which were, in turn,
%% Based on the style files for ACL-2013, which were, in turn,
%% Based on the style files for ACL-2012, which were, in turn,
%% based on the style files for ACL-2011, which were, in turn,
%% based on the style files for ACL-2010, which were, in turn,
%% based on the style files for ACL-IJCNLP-2009, which were, in turn,
%% based on the style files for EACL-2009 and IJCNLP-2008...

%% Based on the style files for EACL 2006 by
%%e.agirre@ehu.es or Sergi.Balari@uab.es
%% and that of ACL 08 by Joakim Nivre and Noah Smith

\documentclass[11pt]{article}
\usepackage{semeval2014}
\usepackage{times}
\usepackage{url}
\usepackage{latexsym}

%\setlength\titlebox{5cm}

% You can expand the titlebox if you need extra space
% to show all the authors. Please do not make the titlebox
% smaller than 5cm (the original size); we will check this
% in the camera-ready version and ask you to change it back.

\newcommand{\wsname}{SemEval-2014}
\newcommand{\submissionpage}{\url{http://alt.qcri.org/semeval2014/index.php?id=cfp}}
\newcommand{\filename}{semeval2014}
\newcommand{\contact}{pnakov qf.org.qa}

\title{SemEval-2014: Instructions for Formatting the Papers}

\author{First Author \\
  Affiliation / Address line 1 \\
  Affiliation / Address line 2 \\
  Affiliation / Address line 3 \\
  {\tt email@domain} \\\And
  Second Author \\
  Affiliation / Address line 1 \\
  Affiliation / Address line 2 \\
  Affiliation / Address line 3 \\
  {\tt email@domain} \\}

\date{}

\begin{document}
\maketitle
\begin{abstract}
  This document contains the instructions for preparing a camera-ready
  manuscript in a two-column format for \wsname{}.
  The document itself
  conforms to its own specifications, and is therefore an example of
  what your manuscript should look like. These instructions should be
  used for both papers submitted for review and for final versions of
  accepted papers.  Authors are asked to conform to all the directions
  reported in this document.
\end{abstract}

\section{Credits}

This document has been adapted from the instructions for the ACL-2014
proceedings compiled by Alexander Koller and Yusuke Miyao,
which are, in turn, based on the instructions for earlier ACL proceedings,
including those for ACL-2012 by Maggie Li and Michael
White, those from ACL-2010 by Jing-Shing Chang and Philipp Koehn,
those for ACL-2008 by Johanna D. Moore, Simone Teufel, James Allan,
and Sadaoki Furui, those for ACL-2005 by Hwee Tou Ng and Kemal
Oflazer, those for ACL-2002 by Eugene Charniak and Dekang Lin, and
earlier ACL and EACL formats. Those versions were written by several
people, including John Chen, Henry S. Thompson and Donald
Walker. Additional elements were taken from the formatting
instructions of the {\em International Joint Conference on Artificial
  Intelligence}.


\section{Summary}

\textbf{NOTE:} The formatting we use for SemEval-2014 is the same as that for *SEM-2014,
except that
(i) the submissions should NOT be anonymized,
(ii) there are restrictions on how the paper can be named,
and
(iii) there are also differences in page lengths.
Note that the style is quite different from COLING-2014,
who use a one-column format!

\section{Introduction}
\label{intro}

%
% The following footnote without marker is needed for the camera-ready
% version of the paper.
% Comment out the instructions (first text) and uncomment the 8 lines
% under "final paper" for your variant of English.
%
\blfootnote{
    %
    % for review submission
    %
    \hspace{-0.65cm}  % space normally used by the marker
    Place licence statement here for the camera-ready version, see
    Section~\ref{licence} of the instructions for preparing a
    manuscript.
    %
    % % final paper: en-uk version (to license, a licence)
    %
    % \hspace{-0.65cm}  % space normally used by the marker
    % This work is licensed under a Creative Commons
    % Attribution 4.0 International Licence.
    % Page numbers and proceedings footer are added by
    % the organisers.
    % Licence details:
    % \url{http://creativecommons.org/licenses/by/4.0/}
    %
    % % final paper: en-us version (to licence, a license)
    %
    % \hspace{-0.65cm}  % space normally used by the marker
    % This work is licenced under a Creative Commons
    % Attribution 4.0 International License.
    % Page numbers and proceedings footer are added by
    % the organizers.
    % License details:
    % \url{http://creativecommons.org/licenses/by/4.0/}
}

The following instructions are directed to authors of papers submitted
to \wsname{} or accepted for publication in its proceedings. All
authors are required to adhere to these specifications. Authors are
required to provide a Portable Document Format (PDF) version of their
papers. \textbf{The proceedings are designed for printing on A4
  paper.}

Authors from countries in which access to word-processing systems is
limited should contact \contact{}
as soon as possible.


\section{Length and Title of Submissions}
\label{sec:length}

\subsection{SemEval-2014 Task Description Paper}

\begin{itemize}
\item \textbf{Paper Title:}  The title should follow the pattern ``SemEval-2014 Task $<$NUM$>$: $<$TASK NAME$>$''.
For example: ``SemEval-2014 Task 3: Cross-Level Semantic Similarity''

\item \textbf{Page Limit:} Up to 8 pages + 2 pages for references
\end{itemize}


\subsection{SemEval-2014 System Description Paper}

\begin{itemize}
\item \textbf{Paper Title:}  The title should follow the pattern ``$<$TEAM\_ID$>$: $<$Paper Title$>$'',
    where $<$TEAM\_ID$>$ is the team ID used when registering with SemEval-2014.

\item \textbf{Page Limit:} A system description paper has up to 4 pages of content + 2 pages for references.
    If a team participates in more than one task, there are two options:
    \begin{itemize}
        \item If the same system (with some modifications) was used for the different tasks, there should be a single system description paper, where each additional task adds + 2 additional pages of content, up to a maximum of 8 pages of content, + 2 pages for references. As START asks to assign a paper to a single task, select the task that fits best.
        \item If the systems used for the different tasks are sufficiently different, participants can submit separate description papers for each task they participated in.
    \end{itemize}

\end{itemize}




\section{General Instructions}

Manuscripts must be in two-column format.  Exceptions to the
two-column format include the title, authors' names and complete
addresses, which must be centered at the top of the first page, and
any full-width figures or tables (see the guidelines in
Subsection~\ref{ssec:first}). {\bf Type single-spaced.}  Start all
pages directly under the top margin. See the guidelines later
regarding formatting the first page.  The manuscript should be
printed single-sided and its length
should not exceed the maximum page limit described in Section~\ref{sec:length}.
Do not number the pages.


\subsection{Electronically-available resources}

We strongly prefer that you prepare your PDF files using \LaTeX{} with
the official \wsname{} style file (\filename{}.sty) and bibliography style
(acl.bst). These files are available at
\submissionpage{}.
You will also find the document
you are currently reading (semeval2014.pdf) and its \LaTeX{} source code
(\filename{}.tex) on this website.

Alternatively, you can use Microsoft Word to produce your PDF file. In
this case, we strongly recommend the use of the Word template file
(\filename{}.dot) on the \wsname{} website. If you have an option, we
recommend that you use the \LaTeX2e{} version.



\subsection{Format of Electronic Manuscript}
\label{sect:pdf}

For the production of the electronic manuscript you must use Adobe's
Portable Document Format (PDF). PDF files are usually produced from
\LaTeX{} using the \textit{pdflatex} command. If your version of
\LaTeX{} produces Postscript files, you can convert these into PDF
using \textit{ps2pdf} or \textit{dvipdf}. On Windows, you can also use
Adobe Distiller to generate PDF.

Please make sure that your PDF file includes all the necessary fonts
(especially tree diagrams, symbols, and fonts with Asian
characters). When you print or create the PDF file, there is usually
an option in your printer setup to include none, all or just
non-standard fonts.  Please make sure that you select the option of
including ALL the fonts. \textbf{Before sending it, test your PDF by
  printing it from a computer different from the one where it was
  created.} Moreover, some word processors may generate very large PDF
files, where each page is rendered as an image. Such images may
reproduce poorly. In this case, try alternative ways to obtain the
PDF. One way on some systems is to install a driver for a postscript
printer, send your document to the printer specifying ``Output to a
file'', then convert the file to PDF.

It is of utmost importance to specify the \textbf{A4 format} (21 cm
x 29.7 cm) when formatting the paper. When working with
{\tt dvips}, for instance, one should specify {\tt -t a4}.

If you cannot meet the above requirements for the
production of your electronic submission, please contact the
publication chairs as soon as possible.


\subsection{Layout}
\label{ssec:layout}

Format manuscripts two columns to a page, in the manner these
instructions are formatted. The exact dimensions for a page on A4
paper are:

\begin{itemize}
\item Left and right margins: 2.5 cm
\item Top margin: 2.5 cm
\item Bottom margin: 2.5 cm
\item Column width: 7.7 cm
\item Column height: 24.7 cm
\item Gap between columns: 0.6 cm
\end{itemize}

\noindent Papers should not be submitted on any other paper size.
If you cannot meet the above requirements for
the production of your electronic submission, please contact the
publication chairs above as soon as possible.


\subsection{Fonts}

For reasons of uniformity, Adobe's {\bf Times Roman} font should be
used. In \LaTeX2e{} this is accomplished by putting

\begin{quote}
\begin{verbatim}
\usepackage{times}
\usepackage{latexsym}
\end{verbatim}
\end{quote}
in the preamble. If Times Roman is unavailable, use {\bf Computer
  Modern Roman} (\LaTeX2e{}'s default).  Note that the latter is about
  10\% less dense than Adobe's Times Roman font.

The {\bf Times New Roman} font, which is configured for us in the
Microsoft Word template (FILENAME.dot) and which some Linux
distributions offer for installation, can be used as well.

\begin{table}[h]
\begin{center}
\begin{tabular}{|l|rl|}
\hline \bf Type of Text & \bf Font Size & \bf Style \\ \hline
paper title & 15 pt & bold \\
author names & 12 pt & bold \\
author affiliation & 12 pt & \\
the word ``Abstract'' & 12 pt & bold \\
section titles & 12 pt & bold \\
document text & 11 pt  &\\
captions & 11 pt & \\
abstract text & 10 pt & \\
bibliography & 10 pt & \\
footnotes & 9 pt & \\
\hline
\end{tabular}
\end{center}
\caption{\label{font-table} Font guide. }
\end{table}

\subsection{The First Page}
\label{ssec:first}

Centre the title, author's name(s) and affiliation(s) across both
columns. Do not use footnotes for affiliations. Do not include the
paper ID number assigned during the submission process. Use the
two-column format only when you begin the abstract.

{\bf Title}: Place the title centred at the top of the first page, in
a 15 pt bold font. (For a complete guide to font sizes and styles,
see Table~\ref{font-table}) Long titles should be typed on two lines
without a blank line intervening. Approximately, put the title at 2.5
cm from the top of the page, followed by a blank line, then the
author's names(s), and the affiliation on the following line. Do not
use only initials for given names (middle initials are allowed). Do
not format surnames in all capitals (e.g., use ``Schlangen'' not
``SCHLANGEN'').  Do not format title and section headings in all
capitals as well except for proper names (such as ``BLEU'') that are
conventionally in all capitals.  The affiliation should contain the
author's complete address, and if possible, an electronic mail
address. Start the body of the first page 7.5 cm from the top of the
page.

The title, author names and addresses should be completely identical
to those entered to the electronical paper submission website in order
to maintain the consistency of author information among all
publications of the conference. If they are different, the publication
chairs may resolve the difference without consulting with you; so it
is in your own interest to double-check that the information is
consistent.

{\bf Abstract}: Type the abstract at the beginning of the first
column. The width of the abstract text should be smaller than the
width of the columns for the text in the body of the paper by about
0.6 cm on each side. Center the word {\bf Abstract} in a 12 pt bold
font above the body of the abstract. The abstract should be a concise
summary of the general thesis and conclusions of the paper. It should
be no longer than 200 words. The abstract text should be in 10 pt font.

{\bf Text}: Begin typing the main body of the text immediately after
the abstract, observing the two-column format as shown in
the present document. Do not include page numbers.

{\bf Indent} when starting a new paragraph. Use 11 pt for text and
subsection headings, 12 pt for section headings and 15 pt for
the title.

{\bf Licence}: Include a licence statement as an unmarked (unnumbered)
footnote in the left column of the first page of the final, camera-ready paper.
See Section~\ref{licence} below for details and motivation.


\subsection{Sections}

{\bf Headings}: Type and label section and subsection headings in the
style shown on the present document.  Use numbered sections (Arabic
numerals) in order to facilitate cross references. Number subsections
with the section number and the subsection number separated by a dot,
in Arabic numerals. Do not number subsubsections.

{\bf Citations}: Citations within the text appear in parentheses
as~\cite{Gusfield:97} or, if the author's name appears in the text
itself, as Gusfield~\shortcite{Gusfield:97}.  Append lowercase letters
to the year in cases of ambiguity.  Treat double authors as
in~\cite{Aho:72}, but write as in~\cite{Chandra:81} when more than two
authors are involved. Collapse multiple citations as
in~\cite{Gusfield:97,Aho:72}. Also refrain from using full citations
as sentence constituents. We suggest that instead of
\begin{quote}
  ``\cite{Gusfield:97} showed that ...''
\end{quote}
you use
\begin{quote}
``Gusfield \shortcite{Gusfield:97}   showed that ...''
\end{quote}

If you are using the provided \LaTeX{} and Bib\TeX{} style files, you
can use the command \verb|\newcite| to get ``author (year)'' citations.

As reviewing will NOT be double-blind, the submitted version of the papers
should include the authors' names and affiliations.
Self-references are perfectly fine as are
acknowledgements, project names, grant numbers, resources, etc.

\textbf{References}: Gather the full set of references together under
the heading {\bf References}; place the section before any Appendices,
unless they contain references. Arrange the references alphabetically
by first author, rather than by order of occurrence in the text.
Provide as complete a citation as possible, using a consistent format,
such as the one for {\em Computational Linguistics\/} or the one in the
{\em Publication Manual of the American
Psychological Association\/}~\cite{APA:83}.  Use of full names for
authors rather than initials is preferred.  A list of abbreviations
for common computer science journals can be found in the ACM
{\em Computing Reviews\/}~\cite{ACM:83}.

The \LaTeX{} and Bib\TeX{} style files provided roughly fit the
American Psychological Association format, allowing regular citations,
short citations and multiple citations as described above.

{\bf Appendices}: Appendices, if any, directly follow the text and the
references (but see above).  Letter them in sequence and provide an
informative title: {\bf Appendix A. Title of Appendix}.

\subsection{Footnotes}

{\bf Footnotes}: Put footnotes at the bottom of the page and use 9 pt
text. They may be numbered or referred to by asterisks or other
symbols.\footnote{This is how a footnote should appear.} Footnotes
should be separated from the text by a line.\footnote{Note the line
separating the footnotes from the text.}

\subsection{Graphics}

{\bf Illustrations}: Place figures, tables, and photographs in the
paper near where they are first discussed, rather than at the end, if
possible.
Wide illustrations may run across both columns.
Colour
illustrations are discouraged, unless you have verified that
they will be understandable when printed in black ink.

{\bf Captions}: Provide a caption for every illustration; number each one
sequentially in the form:  ``Figure 1. Caption of the Figure.'' ``Table 1.
Caption of the Table.''  Type the captions of the figures and
tables below the body, using 11 pt text.


\subsection{Licence Statement}
\label{licence}

To avoid the need for sending signed copyright hand-over forms to the
organisers or publishers, we require that authors license their
camera-ready papers under a
Creative Commons Attribution 4.0 International Licence
(CC-BY).
This means that authors (or their employers) retain copyright but
grant everybody
the right to adapt and re-distribute their paper
as long as the authors are credited and modifications listed,
see
\url{http://creativecommons.org/licenses/by/4.0/} for the
licence terms.

Depending on whether you use American or British English in your
paper, please include one of the following as an unmarked
(unnumbered) footnote on page 1 of your paper.
The \LaTeX{} style file (\filename{}.sty) adds a command
\texttt{blfootnote} for this purpose, and usage of the command is
prepared in the \LaTeX{} source code (\filename{}.tex) at the start
of Section~\ref{intro} ``Introduction''.

\begin{itemize}
    %
    % % final paper: en-uk version (to license, a licence)
    %
    \item  This work is licensed under a Creative Commons
           Attribution 4.0 International Licence.
           Page numbers and proceedings footer are added by
           the organisers.
           Licence details:
           \url{http://creativecommons.org/licenses/by/4.0/}
    %
    % % final paper: en-us version (to licence, a license)
    %
    \item  This work is licenced under a Creative Commons
           Attribution 4.0 International License.
           Page numbers and proceedings footer are added by
           the organizers.
           License details:
           \url{http://creativecommons.org/licenses/by/4.0/}
\end{itemize}


We cannot accept the restrictions ``NonCommercial'' (NC) and
``ShareAlike'' (SA) as they would
limit our options for an on-demand printing service
(we hope to find a partner for this after the conference)
and the restriction ``NoDerivatives'' (ND) as it may impede our
ability to add page numbers and a footer, as well as our ability
to combine papers into volumes.
We intend to publish the proceedings before the conference on
the ACL Anthology and to hand out a memory stick copy at the
conference.
Mentioning in the footnote that we, the organisers, add page numbers
and proceedings footer is a requirement stemming
from the CC-BY licence terms.

%
% We are not doing the following for NAME-OF-WORKSHOP. Maybe next time.
%
% \section{XML conversion and supported \LaTeX\ packages}
%
% ACL 2014 innovates over earlier years in that we will attempt to
% automatically convert your \LaTeX\ source files to machine-readable
% XML with semantic markup. This will facilitate future research that
% uses the ACL proceedings themselves as a corpus.
%
% We encourage you to submit a ZIP file of your \LaTeX\ sources along
% with the camera-ready version of your paper. We will then convert them
% to XML automatically, using the LaTeXML tool
% (\url{http://dlmf.nist.gov/LaTeXML}). LaTeXML has \emph{bindings} for
% a number of \LaTeX\ packages, including the ACL 2014 stylefile. These
% bindings allow LaTeXML to render the commands from these packages
% correctly in XML. For best results, we encourage you to use the
% packages that are officially supported by LaTeXML, listed at
% \url{http://dlmf.nist.gov/LaTeXML/manual/included.bindings}





\section{Translation of non-English Terms}

It is also advised to supplement non-English characters and terms
with appropriate transliterations and/or translations
since not all readers understand all such characters and terms.
Inline transliteration or translation can be represented in
the order of: original-form transliteration ``translation''.


\section*{Acknowledgements}

The acknowledgements should go immediately before the references.  Do
not number the acknowledgements section. Do not include this section
when submitting your paper for review.

% include your own bib file like this:
%\bibliographystyle{acl}
%\bibliography{acl2014}

\begin{thebibliography}{}

\bibitem[\protect\citename{Aho and Ullman}1972]{Aho:72}
Alfred~V. Aho and Jeffrey~D. Ullman.
\newblock 1972.
\newblock {\em The Theory of Parsing, Translation and Compiling}, volume~1.
\newblock Prentice-{Hall}, Englewood Cliffs, NJ.

\bibitem[\protect\citename{{American Psychological Association}}1983]{APA:83}
{American Psychological Association}.
\newblock 1983.
\newblock {\em Publications Manual}.
\newblock American Psychological Association, Washington, DC.

\bibitem[\protect\citename{{Association for Computing Machinery}}1983]{ACM:83}
{Association for Computing Machinery}.
\newblock 1983.
\newblock {\em Computing Reviews}, 24(11):503--512.

\bibitem[\protect\citename{Chandra \bgroup et al.\egroup }1981]{Chandra:81}
Ashok~K. Chandra, Dexter~C. Kozen, and Larry~J. Stockmeyer.
\newblock 1981.
\newblock Alternation.
\newblock {\em Journal of the Association for Computing Machinery},
  28(1):114--133.

\bibitem[\protect\citename{Gusfield}1997]{Gusfield:97}
Dan Gusfield.
\newblock 1997.
\newblock {\em Algorithms on Strings, Trees and Sequences}.
\newblock Cambridge University Press, Cambridge, UK.

\end{thebibliography}

\end{document}
